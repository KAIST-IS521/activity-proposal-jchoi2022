\documentclass[a4paper, 11pt]{article}

\usepackage{kotex} % Comment this out if you are not using Hangul
\usepackage{fullpage}
\usepackage{listings}
\usepackage{hyperref}
\usepackage{amsthm}
\usepackage[numbers,sort&compress]{natbib}
\usepackage[T1]{fontenc} % to fix double-quote problem in lstlisting

\theoremstyle{definition}
\newtheorem{exercise}{Exercise}

\begin{document}
%%% Header starts
\noindent{\large\textbf{IS-521 Activity Proposal}\hfill
                \textbf{Jaeseung Choi}} \\
         {\phantom{} \hfill \textbf{jchoi-2022}} \\
         {\phantom{} \hfill Due Date: April 15, 2017} \\
%%% Header ends

\section{Activity Overview}
이 활동에서는 llvm의 프레임워크를 사용하여, 주어진 C 코드에서 커맨드 인젝션
취약점을 컴파일 타임에 찾아내는 분석기를 만든다. 이 활동을 통해 수강생들은 정적
분석이라는 기술이 어떻게 작동하는지 개략적으로 이해하고, 해당 기술의 한계점에
대해서도 생각해 볼 기회를 갖는다.

프로그램이 점차 크고 복잡해져 가면서, 프로그램의 취약점을 자동으로 탐지하는
기술의 중요성이 높아지고 있다. 사람이 직접 모든 코드를 살펴보고 보안 취약점을
검사하는 것이 힘들어지고 있기 때문이다. 프로그램의 취약점을 자동으로 찾기 위한
여러 방법이 연구되어 왔는데, 정적 분석(static analysis) 또한 그 중 하나이다.

정적 분석은 프로그램을 직접 실행해 보지 않고도 실행 도중에 발생할 수 있는 모든
상황을 안전하게 어림잡는(over-approximate) 기술이다. 이러한 정적 분석의 근간은
요약 해석(abstract interpretation)이라는 기법에 있다\cite{abstract, framework}
요약 해석은 프로그램의 실행 도중에 발생할 수 있는 모든 상황을 발견할 때까지
프로그램을 요약 실행 의미(abstract semantics)에 따라 반복적으로 실행한다. 모든
가능한 상황을 발견했는지는 요약 공간(abstract domain)에서의 고정점(fixpoint)에
도달했는지 확인함으로써 알 수 있다\cite{PALecture}.

커맨드 인젝션 취약점은 사용자의 악의적인 입력이 \texttt{system} 함수의 인자로
사용되어 발생한다. 이 취약점은 바이너리 파일의 실행에 대한 이해 없이도 설명
가능한 취약점이므로 본 과목의 범위를 벗어나지 않는다. 또한 이 취약점은 테인트
분석을 통해 비교적 간단하게 탐지할 수 있기 때문에 활동에서 다루기 적절하다. 외부
입력을 읽어들이는 함수 및 \texttt{argv} 인자를 테인트 소스로, \texttt{system}
함수 호출을 테인트 싱크로 놓은 다음, 소스에서 싱크까지 테인트된 값이 흘러갈 수
있는지를 요약 해석 기술을 활용하여 체크하면 된다.

\texttt{LLVM} 컴파일러가 제공하는 프레임워크에 올라타면 비교적 적은 비용으로
이러한 정적 분석기를 구현하는 것이 가능하다. \texttt{LLVM} 컴파일러에는 주어진
프로그램을 변환하거나 검사할 수 있는 \emph{Pass}라는 기능이 있다\cite{LLVMPass}.
(실제로 \texttt{LLVM}의 최적화 과정은 프로그램을 변환하는 일련의 \emph{Pass}들로
이루어져 있다.) 프로그램의 각 함수에 대해 포맷 스트링 취약점을 검사하는
\emph{Function Pass}를 만들어 기존의 \texttt{LLVM}에 추가하면, 프로그램 파싱
등을 신경쓰지 않고 손쉽게 분석기를 구현할 수 있다. 각 함수마다 별개로 검사를
하므로 구현체는 intra-procedural 방식의 분석기가 될 것이다.


\section{Exercises}

이 활동에서 학생들은 다음과 같은 세부적인 과제들을 수행하게 된다.

\begin{exercise}
 주어진 뼈대코드에서 빈 부분을 채워넣어서 flow-insensitive 버전의 정적 분석기를
완성한다. 뼈대코드에서는 테인트 분석을 위한 도메인을 제공하며, 프로그램을
반복적으로 실행하며 고정점을 체크하는 코드를 제공한다. 수강생들은 \texttt{LLVM
IR}의 각 명령에 대한 요약 실행의미(abstract semantics)를 채워넣기만 하면
분석기가 완성된다. 또한 \texttt{system} 함수에서는 테인트된 값이 인자로
들어왔는지 체크하여 알람을 발생시키는 코드도 작성해야 할 것이다.
\end{exercise}

\begin{exercise}
 이제 앞에서 완성한 분석기를 개선하여 flow-sensitive 버전의 정적 분석기를
만든다. 위에서 만든 flow-insensitive 버전의 분석기는 프로그램의 각 지점을
구분하지 않고 하나의 요약 상태를 구한다. 반면 flow-sensitive 버전의 분석기는
프로그램의 각 지점마다 별개의 요약 상태를 계산하므로 더 정확한 분석 결과를
얻을 수 있다.

 예제 \ref{fig:example}은 flow-insensitive 버전의 정적 분석기는 가짜 알람을
발생시키지만 flow-sensitive 버전의 정적 분석기는 가짜 알람을 발생시키지 않는
프로그램이다. 5번째 줄의 \texttt{system} 함수를 호출하는 시점에서
\texttt{buf}는 사용자 입력값을 저장하고 있지 않으므로, 해당 함수 호출은
안전하다. 그러나 flow-insensitive 버전의 분석기는 프로그램 각 지점을 구분하지
않고 분석하므로, \texttt{buf}의 내용이 테인트되었다고 판단하여 알람을 발생시킬
것이다.

\begin{figure}
\begin{lstlisting}[language=C, frame=single, basicstyle=\footnotesize,
numbers=left, numbersep=1em, xleftmargin=2em, showstringspaces=false]
int main(void)
{
  char buf[BUFSIZE];
  strncpy(buf, "pwd", BUFSIZE);
  system(buf); // Not vulnerable, but flow-sensitive analyzer will raise alarm
  read(0, buf, BUFSIZE);
  ...
}
\end{lstlisting}
 \caption{Flow-sensitive 분석기가 정확히 분석하지 못하는 예제}
  \label{fig:example}
\end{figure}
\end{exercise}

\begin{exercise}
 끝으로, 위에서 만든 flow-sensitive 버전의 정적 분석기조차 제대로 분석하지 못하고
가짜 알람(false positive)을 발생시키는 예제를 수강생이 직접 만들어 본다. 정적
분석은 프로그램의 실행을 어림잡는(over-approximate) 기술이므로, 가짜 알람이
발생하는 것이 불가피하다. 어떤 경우에 분석의 결과가 부정확해지는지 생각하여 예제
프로그램을 직접 작성해 본다.

\end{exercise}

\section{Expected Solutions \& Evaluation}

Exercise 1에서 학생들이 제출한 코드는 LLVM IR의 각 명령에 대해 올바르게 요약
실행의미를 정의해야 한다. 요약 실행의미가 올바르게 정의되었다면, 완성된
분석기는 테스트 프로그램에서 포맷 스트링 취약점을 놓치지 않고 찾을 수 있을
것이다. Flow-insensitive한 분석으로 커맨드 인젝션 취약점을 잘 찾아낼 수 있는
예제들을 여러개 준비하여 채점용 테스트케이스로 활용할 수 있다.

Exercise 2에서 학생들이 작성한 분석기는 더 정확한 분석 결과를 내놓을 수 있어야
한다. 예제 \ref{fig:example}과 같은 프로그램에 대해서 가짜 알람을 발생시키지
않아야 할 것이다. 나아가, 효율적인 구현을 위해 \emph{worklist}\cite{PALecture}를
사용하였다면 추가 점수를 주는 것도 스펙으로 생각해 볼 수 있다.

Exercise 3에서 학생들이 작성한 예제 프로그램은 모범 답안으로 제출된 분석기가
가짜 알람을 발생시키는지 확인함으로써 평가 가능하다. 가짜 알람을 발생시키는
프로그램을 만드는 방법은 매우 다양하며, 학생들의 답안은 그중 어떤 방법을
선택해도 무방하다.

\bibliography{references}
\bibliographystyle{plainnat}

\end{document}
